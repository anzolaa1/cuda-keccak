\chapter{Keccak Overview} \label{chap:keccak-overview} 
Keccak is a family of hash functions that are based on the sponge construction and use as a building block a permutation from a set of 7 permutations.
In this overview, we specify these permutations and the Keccak sponge functions. For more information, and for the reference code, please refer to the Keccak web page.

\section{The Keccak-f permutations}
There are 7 Keccak-f permutations, indicated by Keccak-f[b], where \({b = 25 \times 2^l}\) and
l ranges from 0 to 6. Keccak-f [b] is a permutation over \({s \in Z^b_2}\), where the bits of s are 
numbered from 0 to \({b − 1}\). We call b the width of the permutation.\\

In the sequel, the permutation Keccak-f[b] is described on a state \({a}\) that is a three-dimensional array of elements, namely \({a[5][5][w]}s\), 
with \({w = 2l}\). The mapping between the bits of \({s}\) and those of \({a}\) is \({s[w(5y + x) + z] = a[x][y][z]}\). The expression \({a[x][y][z]}\) with
\({x, y \in Z5}\) and \({z \in Z_w}\) , denotes the bit in position \({(x, y, z)}\). It follows that indexing starts from
zero; expressions in the x and y coordinates should be taken modulo 5 and expressions in the
\({z}\) coordinate modulo \({w}\). Both the [y][z] indices or all
three indices, implying that the statement is valid for all values of the omitted indices.\\

Keccak-f [b] is an iterated permutation, consisting of a sequence of \({nr}\) rounds \({R}\), indexed
with \({i_r}\) from 0 to \({nr − 1}\). A round consists of five steps:\\


	$R = \iota \circ \chi \circ \pi \circ \rho \circ \omega$ ,   with \\


\({	\omega : a[x][y][z] \leftarrow a[x][y][z] + \sum_{y'=0}^4 a[x − 1][y'][z] + \sum_{y'=0}^4 a[x + 1][y ′ ][z − 1] ,} \)\\

$\rho : a[x][y][z] \leftarrow a[x][y][z − (t + 1)(t + 2)/2]$,\\
with t satisfying 0 $\leq$ t $<$ 24 and
$ \left( \begin{array}{cc}
0 & 1 \\
2 & 3 \\
\end{array}
\right)^{t}
\left(
\begin{array}{c}
1 \\
0 \\
\end{array}
\right) =
\left(
\begin{array}{c}
x\\
y\\
\end{array}
\right)
$ in GF$(5)^{2 \times 2}$,\\
or $t = -1$ if $x = y = 0$ \\

$ \pi: a[x][y] \leftarrow a[x'][y'] $, with
$\left( \begin{array}{c}
x \\
y \\
\end{array}
\right) = \left( \begin{array}{cc}
0 & 1 \\
2 & 3 \\
\end{array}
\right)
\left( \begin{array}{c}
x' \\
y' \\
\end{array}
\right)$\\

$\chi: a[x] \leftarrow a[x] + (a[]x+1]+1)a[]x+2$\\

$\iota: a \leftarrow a + RC[i_r]$ \\


The additions and multiplications between the terms are in GF(2). With the exception of
the value of the round constants $RC[i_r]$, these rounds are identical. The round constants are
given by (with the first index denoting the round number)\\

$RC[i_r][0][0][2^j − 1] = rc[j + 7i_r ]$, for all $0 \leq j \leq l$,\\

and all other values of $RC[i_r][x][y][z]$ are zero. The values $rc[t] \in $ GF(2) are defined as the
output of a binary linear feedback shift register (LFSR):\\

$rc[t] = (x^t$ mod $x^8 + x^6 + x^5 + x^4 + 1)$ mod $x$ in GF(2)[x].\\

The number of rounds nr is determined by the width of the permutation, namely,\\

\centering
$n_r = 12 + 2l$
