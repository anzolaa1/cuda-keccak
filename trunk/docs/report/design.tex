\chapter{Design}
\section{Single}
The internal status of the Keccak permutation function is composed by 25 words of 64 bits in a grid 5x5. Since in CUDA the bit to bit operations are rather slow, we decided to use 25 thread, one for every word of the internal state. As described below, every single thread, during the computation, is responsible for calculating the value of a single cell of the matrix, identified by the positions X and Y of the state matrix that are the same of thread in the CUDA thread-grid.
\subsection{Chi}
In the implementation of chi we used a matrix 10x5 of 64 bits words, containing the internal state of the Hash function after the previous step, replicated two times. This because, in CUDA, the modulo operator is rather slow compared with cpu; using a 10x5 matrix the threads that need words out of the first 5 columns of the matrix can safety complete the operations. The operators NOT, XOR and AND have been used normally. The result is written in a new matrix that will be the new internal state.
\subsection{Theta}
In the implementation of Theta the internal state is duplicated in a matrix 5x10 for the same reason described for Chi. Every single thread calculate the C value of its own column so that it is repeated 5 times (one for every thread of the column). This procedure is aimed to avoid using IF-patterns that would break the parallelism between thread. The D matrix is calculated in the same way. For the computation of the ROT matrix we were forced to use shift operators that can decrement the performance. At the end every thread copy its result in the corresponding cell of a new matrix that will be the new internal state.
\subsection{Pi}
The Pi step is implemented using 2 matrix 5x5 with the coordinates X and Y of the new positions that the words will have after the permutation. Every thread read this coordinates and copy its state word in a new matrix, in the position read, that will be the new internal state.
\subsection{Rho}
Like in the Theta, in Rho we were forced to used shift operators to implement the bit word rotation; the offset of the rotation depends on the position of the word to rotate in the internal state and is loaded as constant in a 5x5 matrix. The single thread read the offset value and make the rotation of its own word and then copies the result in a new matrix.
\subsection{Iota}
The implementation of Iota is obviously the more simple since in Iota there is only a bit to bit xor between the first word of the internal state and a 64 bits constant different in every round. Those round constant are preloaded and the operators has been used normally.

\section{Multi}

